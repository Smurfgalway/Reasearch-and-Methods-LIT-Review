
\documentclass[conference]{IEEEtran}
\usepackage{blindtext, graphicx}

\ifCLASSINFOpdf

\else

\fi

\hyphenation{op-tical net-works semi-conduc-tor}


\begin{document}
%
% paper title
% can use linebreaks \\ within to get better formatting as desired


\title{The Societal Impact of Artificial Intelligence and what it presents}



\author{\IEEEauthorblockN{Stephen Murphy, G00323639, BSC HONS}}



\maketitle


\begin{abstract}
In this paper will take a in-depth look at how Artificial Intelligence(AI) surrounds us in today's society and the future the adoption of this technology presents. Looking at the past and where it all began to examining present day marvels modern AI has produced. Examining the effects on education, economy, medicine and everyday life. Along with taking a look at the vast benefits we gain from it and the dangers it presents. 
\end{abstract}

\begin{IEEEkeywords}
Artificial, Intelligence, Machine, Learning,  
\end{IEEEkeywords}






\IEEEpeerreviewmaketitle



\section{Introduction}
The concept of artificial intelligence is not a new one but has in fact existed for centuries. As elegantly put by Author Pamela McCorduck the early concept of Artificial intelligence was "an ancient wish to forge the gods"\cite{McCorduck}. The field of study in Artificial intelligence was established in the 1950s with the likes of Alan Turing, John McCarthy, Marvin Minsky and Allen Newell spearheading it. Each contributing key concepts that are touted as the pillars of Artificial Intelligence. Alan Turing is often considered the father of AI. and modern computer science with his contributions coming before the field of study was formally established. Still to this day the Alan Turing's aptly named "Turing Test" \cite{Turing} is used when testing new AI. It wasn't till 1955 that artificial intelligence was established as a formal field of study. John McCarthy coining the term Artificial Intelligence in his proposal for a research project to be studied the following summer to take place in Dartmouth College, Hanover, New Hampshire \cite{Dartmouth}. Meeting in the summer of 1956 for what became know as the Dartmouth Workshop McCarthy meeting with the co-writers of the proposal, Marvin Minisky, Nathaniel Rochester and Claude Shannon. A long with several other intellectual peers beginning a 2 month long study into the topics purposed such as automation, neural networks, a computers ability to use language and self improvement. This workshop is looked to as the event ushering in this new field of study. Since then there have been many revolutionary steps and discoveries made in the field. From there on artificial intelligence has gone through a lot crucial exhaustive study with developments advancing at a quick pace and little stagnation or signs of slowing. AI has become a integral part in many aspects of society. From a creative stand point you can look to how the idea of AI has influenced pop-culture heavily with countless films, literature, television series, and video games being produced or center around the topic to massive audiences. Artificial Intelligence is not some niche topic that only those in the technological circles know or talk about but a almost rudimentary part of the global vocabulary. We have gone from the likes of IBM's Deep Blue a chess playing AI that was taught to play chess to Google Deepmind's AlphaGo a AI designed for general purpose whose self taught success saw it beat the world champion of GO in 2016. Of course talking about such developments seems very impersonal and not very accessible to the average person but that's where we must observe our surroundings and realize AI is everywhere. Each and everyday people interact with AI often without even realizing it. From customer support chatterbots that dealing with consumer complaints to emergency service AI planning life saving routes to helpful assistants in our smart phones telling us the weather forecast Artificial intelligence is a integral part of the world we live in. So now we should examine how exactly we are being affected by this technology and what future it is leading us into. 

\section{Impact}
In today's climate of a technological golden age another industrial revolution almost seems imminent. With parts becoming more readily available everyday, cost of production being reduced year by year and technological leaps. Living in a era like this it is easy to take this all for granted and not take much notice as to how much the world is being shaped by this technology. We have super computers in our back pockets with native AI there to feed us information be it the weather forecast, best places to eat, if a film is worth seeing etc. It is something we have grown accustomed to and for our children and generations to come AI will be a necessary part of life. So with so much information at you finger tips and a technology with a ever expanding level of potential we must examine what we are currently doing with this and where it is leading.
\subsection{Education}
Education is important when talking about any topic. We are always striving to stay as educated as possible and pass on as much knowledge as possible. When relating to AI it is important to both stay educated on the over all technology and topic. Understanding the benefits but also problems it presents. Along with understanding how to use it correctly and effectively. Of course we can't stay on top of every update or upgrade going on as there are so many but it is important to be well read and in the loop. When talking about education we must also look at what AI can do for education and how it is currently being used for educational purposes. Teaching techniques are changing along with the in class experience and how we interact with the information provided. Education is still very much a lecturer or teacher of some kind providing information for students to learn and reproduce. Though with information so readily available it is more so becoming not what you know but how you plan to use the information to better yourself and the field you are working in. Our Teachers are now being accompanied with AI helpers who helpers who both provide an aid to the teacher when teaching but also gathering vital information. These educational AIs are taking account of everything from class attendance to test scores to class participation levels. Recording this information is vital as it provides a look into where changes need to be made and what learning strategies are working. This information also makes it personal to each student as the recorded data is a portfolio of how exactly that person learns and what specific difficulties they face \cite{Education}. 
\subsection{Economy}
Employment and the Jobs market have always been very volatile. We have seen how new technologies can completely change whole industries both creating and eliminating numerous jobs. The industrial revolution of the 1800s ushered in mass production and created employment unlike the world had ever seen. We face such a revolution in modern day as we see with more automation and AI more jobs are becoming redundant. Then need for people to adapt and up skill becomes necessary for what the job market demands in today's world. Companies are opting for AI replacement and automation as a means to cut costs on wages and reducing risk for those in hazardous lines of work. Smarter AI presents a better prospect for carrying out certain jobs as it takes out human error. Though like with all technology there will new jobs created that didn't exist before. Maintenance of these AI will be necessary. It may seem that it is a jobless future with AI taking up all employment but this is not the case there will as be a need for a human perspective. More so the future is looking more like Humans being assisted and working along side AI to shape a better future and solve problems that were previously considered unsolvable. AI also offers new tools to employers when looking to hire. AI equipped with screening algorithms collect the data of a potential employee parsing through everything from work history to social media posts to get a gauge on the personality and work rate of this person \cite{Employers}. This of course puts even more power in the employers hand and makes you conscious of your digital footprint and what data you are sharing with the world. 
\subsection{Medical}
In the medical sector every new technology presents better ways to save lives and help fight illness and diseases. We want our doctors as well equipped as possible when it comes to dealing with something as crucial as human well-being. This is where AI offers a vital helping hand from helping diagnose patients to reading detailed x-rays and scans. Error is a part of the human process and when dealing with people's health and lives the margin for error is minuscule. This why the introduction of AI at the forefront of medical advancements. There has been a study for a proposal of bring in AI as a multipurpose team member for medical professionals. This AI would offer a major reduction in errors made along with offering helpful tactics and reminders to the medical professionals it works alongside \cite{AIDoc}. These developments are promising and also present the idea of humans and AI working together for better solutions. It is important to emphasis that as the future is not one or the other but humans and AI working together. When dealing with patients there will always be a need to human interaction. When it comes to reading complex medical data sets and analyzing medical data AI presents hugely beneficial results helping Identify who is most at risk \cite{atRisk}. AI will be intricate in the future of Modern Medicine. 
\subsection{Everyday Life}
Our everyday lives have been changed and formed by the technology around us. Our interactions with the world and each-other have totally changed in the past decade due to advancements in technology. The likes of internet and social media has opened a whole new world of interaction, revenue streams and revolutionized communication. AI is the next step of technology changing our behaviour and how we live our lives. Our smart phones now come equipped with their own AI be it Apple's Siri, Microsoft's Cortana, or Android's Google Now each offering the user a helping hand and new ways to interact with their devices. On the surface these AI seem very rudimentary but with every update and new interaction they learn more and increase in complex interaction. Initially they were mostly speech to text search but as time has gone on these AI have learned to process more complex questions and actions. Everything from offering detailed weather reports to tweeting for you these AI offer themselves as malfunction assistants in your device. As time passes these AI learn to adapt to accents, follow patterns and take account of your habits and phone activity to offer convenience to their users. We then must look at how this changes use the users as we take advantage of this technology it changes the way we operate. If there is a easier way to do things people will always do so. If instead of reading a 10 page film review a person can ask their device simply if the film is worth seeing and the AI returns a overall review score they will do so. It is not just our smart phones either but our home life as well with the introduction of Amazon's Echo or Google you can control things throughout your house with a simple voice command. These AIs connect to your WIFI network and can do everything from play music to changing temperature of the room. The dream of your own robot butler is becoming a reality and is affordable. Changing our behavior once again as why preform menial tasks like turning off the lights when you can just ask your home AI to do so \cite{SmartHome}. As these devices learn and as we grow accustomed to them they take on their own personalities whether developer given or through user attachment. Amazon personify the Echo by giving it the name Alexa giving the users a friendly help assistant to interact with. As we personify AI we become attached to it, it becomes a part of the family. It is a new form of entertainment and in some cases just something or rather someone to talk to \cite{Personality}. This presents a interesting look into the impact of how technology can change the way we operate day to day. Looking outside the home another huge revolution in AI is self driving cars. From Google's self driving car experiment to Tesla Inc's cars offering a early version of auto pilot. Self driving cars are no longer just a concept but a reality that could see mass production and adoption in the near future. According to the Association For Safe International Road Travel(ASIRT) it is estimated that close to 1.3 million people die each year world wide from road traffic accidents with a further 20 to 50 million injured or left disabled \cite{ASIRT}. Statistics provided by RSA show there have been 127 road fatalities on Irish roads from January to October 31st \cite{RSA}. Taking this information into account we can recognize how dangerous driving is and how human error can play a huge part in it. This is where self driving cars can offer a huge improvement with AI trained to obey the rules of the road and taking out the distractions a normal motorist would face. Self driving cars will not be speeding or checking texts while driving, they will be serving their sole purpose to drive and do so safely. With this though many things must be taken into consideration such as what should a self driving car do in the case of a unavoidable accident. These AI must take into account what situation presents the least fatalities and what constitutes the most importance. It is both a question of morality and safety. This is where the trolley problem presents itself along with other algorithms to help the AI make the most ethical choice. The trolley problem is as follows: There is a runaway trolley barreling down the railway tracks. Ahead, on the tracks, there are five people tied up and unable to move. The trolley is headed straight for them. You are standing some distance off in the train yard, next to a lever. If you pull this lever, the trolley will switch to a different set of tracks. However, you notice that there is one person tied up on the side track. You have two options: 
Do nothing, and the trolley kills the five people on the main track. 
Pull the lever, diverting the trolley onto the side track where it will kill one person.
Which is the most ethical choice? 
So in the case of a unavoidable accident self driving cars become a applied form of the trolley problem \cite{Trolley Problem}. Taking out the human element in driving will reduce road deaths but then in the case of something unavoidable do we want a AI making the decision for us? If so then who is to blame ? and what decision will this AI make. This of course will have to lead to massive changes in insurance laws along with huge reviews by ethics committees \cite{Law}. Self driving cars presents a exciting future but to those in the transport industry may mark the start of unemployment. Of course it will take years for world wide adoption and availability but a self driving future is almost certain. 

\section{Weighing the Odds}
When examining the impact and adoption of AI we must look at how it is effecting us both positively and negatively. It is important to know where we stand currently and how to proceed into the future. We must weigh the pros and cons and see how much we are willing to give up in order to take full advantage of this technology. Do the benefits out weigh the risks?
\subsection{Beneficial Effects}
AI offers a whole range of benefits to society. AI can offer everything from companionship to life saving solutions to unfathomable boast in productivity. A reduction of mortality rates is a certainty with smarter AI being introduced. Hazardous jobs can be now machine operated reducing risks massively and taking people out of danger. As discussed before self driving cars being introduced providing a great service and reducing human error on roads \cite{Trolley Problem}. The need to learn to drive will not be necessary and roads will be safer. Smart homes being introduced giving use the tools to completely up the productivity of our homes and creating a new interactive home environment \cite{SmartHome}. Medical use of AI is and will save lives. Offering new ways to approach health problems and offering new solutions previously not available \cite{atRisk}. Huge leaps in advancements and production will be and are being produced with AI. Better and more targeted education will be provided through the use of AI. Training in a whole new generation equipped with a wide range of skills to build a better future \cite{Education}. 
\subsection{Dangers and Controversy}
AI is not without its dangers and share of controversy. Data protection in this modern world is a huge issue with companies and governments guiltily of spying on and using peoples private data. Companies sell information to advertisers which create target ads for the consumers. AI will contribute to this as it will have all your data and having observed your habits know what you want and need. Home AI as previously discussed offer great benefit but also some worrying prospects. Though voice commanded these AI are always on and always listening so it is hard to know what exactly is and isn't being tracked. Amazon echo records 60 seconds before wake word. Amazon is not currently selling your audio or data to anyone. Google on the other hand shares data recorded with developers. This sensitive information in the wrong hands could be very damaging. Nothing is exempt from hacking which presents a huge issue. Hackers can get their hands on information such as how frequently you leave your house, your search history and so much more \cite{SmartHome}. As discussed previously employers are using AIs to vet potential employees. This will leave a lot of people starting with a disadvantage as with the age of social media everything you've posted is out there. If there is anything incriminating or negative about you on social media or the internet it will be found \cite{Employers}. It is human nature to make mistakes and especially with younger people unfortunate and ignorant things will be said. So do we judge someone one what they said/posted years ago? It becomes a very grey area and morality comes into play. There is definitely a question of morality and ethics when it comes to self driving cars as discussed earlier. Do we let these AI's decide what is most important and who to save in the case of a accident. Can these cars be hacked? If so cars become weaponized in the hands of these hackers. There is a lot to dwell on with the introduction of AI into society. We must also ask the question as to how much power we can trust it with and if there is a possibility of sentience that could present a danger to us as a whole or if it is just paranoia. Human rejection can also present problems with fears clouding judgment and refusal to accept new ways of doing things \cite{Dangers}.

\section{Conclusion}
Artificial Intelligence is a fascinating technology with endless possibilities. It very much is the future of a lot of industries and already has a grip on society and how we function. The seeds have been planted for some time now and we are beginning to reap the rewards and see what exactly we have created. AI is coming out of its infancy and getting ready to take on the world but are we ready for it? From examining it in depth I think we should be very aware of how we interact with this technology and each other. How our actions and social ability is being shaped by AI around us and being mindful to not become too dependant on it. I feel it is very important to stay education on that matter and to be able to identify the benefits and dangers it presents. Not to live in fear of it or be swept up in the hype of it but be balance and somewhere in the middle. It is important to not be very black and white when discussing this topic but allow for the many shades of grey that encompass this advancement. AI will be in our future whether we like it or not it is how we use it and work with it that is important. I truly believe working along side AI for a better future is the way forward. "An ancient wish to forge the gods"\cite{McCorduck} this wish has very much become a reality and is only going to develop and advance more in years to come. So let us take this society shaping technology we have today and harness it for a better tomorrow.

% use section* for acknowledgement
\section*{Acknowledgment}

I would like to thank my family. My mother in particular as without her undying support and encouragement I would have not made it to this point in my college career. I would also like to thank my peers for providing a friendly learning environment and help me work and go over things I hadn't fully grasped. I would to thank my lecturers for a fantastic learning experience and passing on their vital knowledge to get me to where I am going.
\ifCLASSOPTIONcaptionsoff
  \newpage
\fi


\begin{thebibliography}{1}

\bibitem{McCorduck}
Pamela McCorduck, \emph{Machines who think: a personal inquiry into the history and prospects of artificial intelligence.} \emph{2 edition Published March 19, 2004.} \emph{Available for purchase at https://www.amazon.com/Machines-Who-Think-Artificial-Intelligence/dp/1568812051}

\bibitem{Turing}
A. M. Turing, \emph{Computing Machinery and Intelligence} \emph{Vol. 59, No. 236 published Oct., 1950.}

\bibitem{Dartmouth} 
John McCarthy, Marvin Minsky, Claude Elwood Shannon, Nathaniel Rochester, \emph{ A Proposal for the Dartmouth Summer Research Project on Artificial Intelligence} \emph{https://www.aaai.org/ojs/index.php/aimagazine/article/view/1904} found on Google Scholar

\bibitem{Education}
DICKSON, BEN 2017, \emph{ 'HOW ARTIFICIAL INTELLGENCE IS SHAPING THE FUTURE OF EDUCATION', PC Magazine, pp. 105-115, Business Source Complete, EBSCOhost,} 

\bibitem{Employers}
Captain, Sean. "NOT-SO-HUMAN RESOURCES." Fast Company no. 209 (October 2016): 44. Business Source Complete, EBSCOhost (accessed December 1, 2017)

\bibitem{AIDoc}
Buzaev, I, Plechev, V, Nikolaeva, I, & Galimova, R 2016, 'Original Article: Artificial intelligence: Neural network model as the multidisciplinary team member in clinical decision support to avoid medical mistakes', Chronic Diseases And Translational Medicine, 2, pp. 166-172, ScienceDirect, EBSCOhost, viewed 2 December 2017.

\bibitem{SmartHome}
PITSKER, K 2017, 'HOME SMART HOME', Kiplinger's Personal Finance, 71, 10, pp. 64-69, Business Source Complete, EBSCOhost, 

\bibitem{Personality}
Purington, Amanda & G. Taft, Jessie & Sannon, Shruti & Bazarova, Natalya (Natalie & Hardman Taylor, Samuel. (2017). "Alexa is my new BFF": Social Roles, User Satisfaction, and Personification of the Amazon Echo. 2853-2859. 10.1145/3027063.3053246. 

\bibitem{ASIRT}
http://asirt.org/initiatives/informing-road-users/road-safety-facts/road-crash-statistics

\bibitem{RSA}
http://www.rsa.ie/RSA/Road-Safety/Our-Research/Deaths-injuries-on-Irish-roads/

\bibitem{Trolley Problem}
Nyholm, S, & Smids, J 2016, 'The Ethics of Accident-Algorithms for Self-Driving Cars: an Applied Trolley Problem?', Ethical Theory & Moral Practice, 19, 5, pp. 1275-1289, Academic Search Complete, EBSCOhost,

\bibitem{Law}
Vellinga, NE 2017, 'From the testing to the deployment of self-driving cars: Legal challenges to policymakers on the road ahead', Computer Law & Security Review: The International Journal Of Technology Law And Practice, 33, pp. 847-863, ScienceDirect, EBSCOhost, 

\bibitem{atRisk}
Nadimpalli, M., 2017. Artificial Intelligence Risks and Benefits. Artificial Intelligence, 6(6).

\bibitem{Dangers}
Dubhashi, D. and Lappin, S., 2017. AI dangers: imagined and real. Communications of the ACM, 60(2), pp.43-45.
\end{thebibliography}

\begin{IEEEbiography}[{\includegraphics[width=1in,height=1.25in,clip,keepaspectratio]{picture}}]{John Doe}
\blindtext
\end{IEEEbiography}

\end{document}


